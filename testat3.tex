% coding:utf-8
\section{Polynomfunktion}
Punkt mit lokalem Extremwert: $(e,g)$\\
Punkt mit Wendepunkt: $(h,i)$
\[f(x) = a \cdot x^3 + b \cdot x^2 + c \cdot x + d\]
\[f'(x) = 3 \cdot a \cdot x^2 + 2 \cdot b \cdot x + c\]
\[f''(x) = 6 \cdot a \cdot x + 2 \cdot b\]
\[f'''(x) = 6 \cdot a\]
\[f(e) = g\]
\[f'(e) = 0\]
\[f''(e) \neq 0\]
\[f(h) = i\]
\[f''(h) = 0\]
\[f'''(h) \neq 0\]
Solve mit TR:
\[\text{solve}\left(\begin{matrix}
a \cdot e^3 + b \cdot e^2 + c \cdot e + d = g\\
3 \cdot a \cdot e^2 + 2 \cdot b \cdot e + c = 0\\
6 \cdot a \cdot e + 2 \cdot b \neq 0\\
a \cdot h^3 + b \cdot h^2 + c \cdot h + d = i\\
6 \cdot a \cdot x + 2 \cdot b\\
6 \cdot a \neq 0
\end{matrix} \quad ,a,b,c,d\right)\]
\[f(x) = a \cdot x^3 + b \cdot x^2 + c \cdot x + d\]

\section{Kettenlinie}
Aufgabenstellung: 
\[y(x) = a \cdot \cosh\left(\frac{x}{c}\right) + b\]
a)\\
\[h = H - y(x) = y(\ell) - y(x)\]
\[h = a \cosh\left(\frac{\ell}{c}\right) + b - \left(a \cosh\left(\frac{x}{c}\right) + b\right)\]
\[h = a \cdot \cosh\left(\frac{\ell}{c}\right) - a \cdot \underbrace{\cosh\left(\frac{x}{c}\right)}_1\]
\[\rightarrow x = 0\]
\[\Rightarrow h_{max} = a \cdot \cosh\left(\frac{\ell}{c}\right)-a\]
\[\underline{\underline{h(\ell) = a \cosh\left(\frac{\ell}{c}\right) - a}}\]
b)
\[h(\ell_1) =a \cdot \cosh\left(\frac{\ell}{c}\right) - a\]
\[\cosh\left(\frac{\ell}{c}\right) = \frac{h(\ell_1 + a}{a}\]
\[\Rightarrow \underline{\underline{\arccosh\left(\frac{h(\ell_1 + a}{a}\right) \cdot a}}\]
c)
\[m(\ell) = h'(\ell) = \sinh\left(\frac{\ell}{c}\right)\]
\[m(\ell) = \tan(\beta) \quad\]
\[\Rightarrow \underline{\underline{\alpha = \frac{\pi}{2} - \beta = \frac{\pi}{2} - \arctan(m(\ell)) = \frac{\pi}{2} - \arctan\left(\sinh\left(\frac{\ell}{c}\right)\right)}}\]

\section{Maximale Fläche}
Aufgabenstellung: \\
\includegraphics[width=0.8\textwidth]{bilder/maximale_flaeche.pdf}
\[a \cdot x^2 + b \cdot y^2 = c\]
Lösung:
\[A = 4 \cdot A_1\]
\[A_1 = x \cdot f(x)\]
\[\Rightarrow A = 4 \cdot x \cdot f(x)\]
% \[f(x) \stackrel{?}{=} \]
\[c = a \cdot x^2 + b \cdot y^2\]
\[\Rightarrow y = \left( \frac{c - a \cdot x^2}{b} \right)^{\frac{1}{2}}\]
\[A = 4 \cdot x \cdot \left( \frac{c - a \cdot x^2}{b} \right)^{\frac{1}{2}}\]
Wo ist $A_1$ maximal? \\
$\rightarrow$ Dort wo die Ableitung $0$ ergibt. 
\[A' \stackrel{!}{=} 0\]
\[A' = 4 \left(\frac{c - a \cdot x^2}{b}\right)^\frac{1}{2} + 4 \cdot x \left(\frac{1}{2} \cdot \left(\frac{c - a \cdot x^2}{b}\right)^{-\frac{1}{2}} \cdot \left(\frac{-2 \cdot a \cdot x}{b}\right)\right) = 0\]
\[A' = \left(\frac{c - a \cdot x^2}{b}\right)^{\frac{1}{2}} + x \cdot \left(\frac{1 \cdot \left(\frac{-2ax}{b}\right)}{2 \cdot \left(\frac{c - ax^2}{b}\right)^{\frac{1}{2}}}\right) = 0\quad \left|\text{mit }\left(\frac{c - ax^2}{b}\right)^{\frac{1}{2}} \right.\text{ erweitern}\]
\[A' = \frac{c - ax^2}{b} + x \cdot \frac{-ax}{b} = 0\]
\[A' = c - ax^2 - ax^2 = c - 2 \left(ax^2\right) = 0\]
\[c = 2 \left(ax^2\right)\]
\[\underline{x = \pm\sqrt{\frac{c}{2a}}}\]
$\rightarrow$ Weil die Fläche nur positiv sein kann, gilt nur $x \leq 0$
\[\Rightarrow A = 4 x \cdot f(x) = 4 x \left(\frac{c - ax^2}{b}\right)^{\frac{1}{2}}\]
\[A = 4 \sqrt{\frac{c}{2a}} \left(\frac{c - a \sqrt{\frac{c}{2a}}^2}{b}\right)^{\frac{1}{2}}\]
\[A = 4 \sqrt{\frac{c}{2a} \left(\frac{c - a \frac{c}{2a}}{b}\right)}\]
\[\underline{\underline{A = 2 \sqrt{\frac{c^2}{ab} } = \frac{2c}{\sqrt{ab}}}}\]

\section{Statue}
Aufgabenstellung: \\
\includegraphics[width=0.8\textwidth]{bilder/statue.pdf}
\[ b = \text{Erde zu Statuenfuss} \]
\[ a = \text{Satuenfuss zu Statuenkopf} \]
\[ d = \text{Statue zu Betrachter} \]
\[ \alpha = \text{Winkel von Statuenkopf zu Statuenfuss} \]
Wo ist der Winkel $\alpha$ maximal? Dort wo die Ableitung der Funktion $\alpha(d)$ Null ergibt also $\alpha'(d)=0$ ist. 
Um dies zu bestimmen muss $\alpha$ definiert werden. 
Da dies auf Anhieb nicht möglich ist, kann man sich folgende Überlegung machen:
\[ \beta = \text{Winkel Betrachter zu Statuenboden} \]
\[ \gamma = \text{Winkel Betrachter zu Statuenkopf} \]
\[ \Rightarrow \gamma = \alpha + \beta \]
\[ tan(\gamma) = \tan(\alpha + \beta) = \left(\frac{a+b}{d}\right) \]
\[ \Rightarrow \alpha + \beta = arctan\left(\frac{a+b}{d}\right) \]
Nun haben wir eine neue Unbekannte $\beta$. Diese muss eliminiert bzw. substituiert werden durch etwas bekanntes oder gesuchtes.
\[ \tan(\beta) = \left(\frac{b}{d}\right) \rightarrow \beta = arctan\left(\frac{b}{d}\right) \]
\[ \Rightarrow \alpha = arctan\left(\frac{a+b}{d}\right) - \beta = arctan\left(\frac{a+b}{d}\right) - arctan\left(\frac{b}{d}\right) \]
\[ \alpha' \stackrel{!}{=} 0 \rightarrow \alpha' = \frac{-(a+b)}{d^2 + (a+b)^2} + \frac{b}{d^2 + b^2} = 0 \]
\[ \Rightarrow d = \sqrt{ab + b^2} \]

\section{Implizites Ableiten}
\[ x^3 + y^3 - a \cdot x \cdot y = 0 \]
a)\\
x und y in gegebene Formel einsetzen. Diese muss 0 werden, damit der entsprechende Punkt auf der Kurve liegt. 
b)\\
\[ x^3 + y^3 - a x y = 0 \]
\[ y = y(x) \quad \text{$y$ mit $y(x)$ substituieren} \]
\[ \rightarrow x^3 + y(x)^3 - a x y(x) = 0 \]
Ableiten mit Kettenregel. $y(x)$ ist jeweils die innere Ableitung. 
\[ \frac{dy}{dx} = 3x^2 + 3y(x)^2 \cdot y'(x) - a\left(y(x) + x \cdot y'(x)\right) = 0 \]
\[ 3x^2 + 3y(x)^2 \cdot y'(x) - a \cdot y(x) - a \cdot x \cdot y'(x) = 0 \]
\[ 3y(x)^2 \cdot y'(x) - a \cdot x \cdot y'(x) = - 3x^2 + a \cdot y(x) \quad \left|- 3x^2 + a \cdot y(x)\right.\]
\[ y'(x) \cdot \left(3y(x)^2 \cdot - a \cdot x \right) = a \cdot y(x) - 3x^2 \quad |y'(x)\text{ ausklammern}\]
\[ y'(x) = \frac{a \cdot y(x) - 3x^2}{3y(x)^2 \cdot - a \cdot x} \quad |:3y(x)^2 \cdot - a \cdot x \]
\[ y' = \frac{a \cdot y - 3x^2}{3y^2 \cdot - a \cdot x} \quad |y(x) = y \]
c)\\
\[ T(x) = f'(x) \cdot (x - x_0) - f(x_0) \]
\[ T(x) =  \frac{a \cdot y_0 - 3x_0^2}{3y_0^2 \cdot - a \cdot x_0} \cdot (x - x_0) - (x_0^3 + y_0^3 - a \cdot x_0 \cdot y_0) \]
$x_0$ und $y_0$ aus Aufgabenstellung einsetzen. \\\\
{\Huge Achtung! \\Fehler in Lösung. \\Wenn du den Fehler findest: \\Mail an: \href{mailto:daniel.winz@stud.hslu.ch}{daniw}}\\
d)\\
\[ f'(0) = 0 \]
\[ \Rightarrow  \frac{a \cdot y - 3x^2}{3y^2 \cdot - a \cdot x} = 0 \]
\[ x^3 + y3 - a \cdot x \cdot y = 0 \]
2 Gleichungen, 2 Unbekannte: Solve mit TR. \\
\[ \Rightarrow x = a \cdot \frac{\sqrt[3]{2}}{3}\]
\[ \Rightarrow y = a \cdot \frac{\sqrt[3]{2^2}}{3}\]