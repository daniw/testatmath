% coding:utf-8
Die Musterlösungen vom Testat 2 sind teilweise noch nicht überprüft und können daher noch Fehler enthalten! \\
Überprüfte Lösungen sind an der doppelten Unterstreichung zu erkennen. 
\section{Gleichung Sekante}
\[ A = (a_x | a_y) \]
\[ B = (b_x | b_y) \]
\[ y = \frac{b_y - a_y}{b_x - a_y} \cdot x + f(0) \]

\section{Gleichung Tangente}
\[ A = (a_x | a_y) \]
\[ f(x) = a x^2 + b x + c \]
\[ T(x) = f'(x_0)(x - x_0) + f(x_0) \]
\[ T(a_x) = (2 a x + b)(x - a_x) + a x^2 + b x + c \]
\[ T(a_x) = 2 a x^2 - 2 a x a_x + b x - b a_x + a x^2 + b x + c \]
\[ T(a_x) = 3 a x^2 - 2 a x a_x + 2 b x - b a_x + c \]

\section{Senkrechter Wurf}


\section{Schnitt mit Gerade}


\section{Intervall einer Funktion}
a) 
Die Funktion ist dort wachsend, wo die Ableitung $\geq 0$ ist. \\
b) 
Die Funktion ist dort rechtsgekrümmt, wo die zweite Ableitung $\leq 0$ ist. 

\section{Unbekannte Funktion Ableitung}
\[ f(x) = \frac{u(x) + a}{u(x) + b} \]
\[ u(0) = c \]
\[ u'(0) = d \]
\[ \underline{\underline{f(0) = \frac{c + a}{c + b}}} \]
\[ f'(0) = \frac{(d) \cdot (c + b) - (c + a) \cdot (d)}{(d + b)^2} \]
\[ f'(0) = \frac{(c d + b d) - (c d + a d)}{(c + b)^2} \]
\[ f'(0) = \frac{c d + b d - c d - a d}{(c + b)^2} \]
\[ \underline{\underline{f'(0) = \frac{b d - a d}{(c + b)^2}}} \]

\section{Differenzierbarkeit}
\[ f_1(x) = \frac{x + a}{a \cdot x} \quad x \leq -1 \]
\[ f_2(x) = \frac{b \cdot x}{x + 2 a} \quad x > -1 \]
\[ f_1(c) = f_2(c) \]
\[ f_1'(c) = f_2'(c) \]
\[ \frac{x + a}{a \cdot x} = \frac{b \cdot x}{x + 2 a} \]
\[ \frac{(x + a)(x + 2 a)}{a \cdot x} = b \cdot x \]
\[ \frac{(x + a)(x + 2 a)}{a \cdot x^2} = b \]

\[ \frac{1 \cdot a \cdot x - (x + a) \cdot a}{(a x)^2} = \frac{b \cdot (x + 2 a) - b x \cdot 1}{(x + 2 a)^2} \]
\[ \frac{a x - a x - a^2}{(a x)^2} = \frac{b x + 2 a b - b x}{(x + 2 a)^2} \]
\[ -\frac{a^2}{(a x)^2} = \frac{2 a b}{(x + 2 a)^2} \]
\[ -\frac{1}{x^2} = \frac{2 a b}{(x + 2 a)^2} \]
\[ -\frac{1}{x^2} = \frac{2 a \frac{(x + a)(x + 2 a)}{a \cdot x^2}}{(x + 2 a)^2} \]
\[ -\frac{1}{x^2} = \frac{2 a \frac{(x + a)}{a \cdot x^2}}{(x + 2 a)} \]
\[ -1 = \frac{2 (x + a)}{(x + 2 a)} \]
\[ -(x + 2 a) = 2 (x + a) \]
\[ -x - 2 a = 2 x + 2 a \]
\[ -3 x = 4 a \]
\[ \underline{\underline{a = -\frac{3}{4} x}} \]
\[ b = \frac{(x + a)(x + 2 a)}{a \cdot x^2} \]
\[ b = \frac{(x -\frac{3}{4} x)(x - \frac{6}{4} x)}{-\frac{3}{4} x \cdot x^2} \]
\[ b = -\frac{(\frac{1}{4} x)(- \frac{1}{2} x)}{\frac{3}{4} x \cdot x^2} \]
\[ b = \frac{\frac{1}{8} x^2}{\frac{3}{4} x \cdot x^2} \]
\[ \underline{\underline{b = \frac{1}{6 x} }} \]
\[  \]
\[  \]

\section{Ableitungsregeln I}


\section{Höhere Ableitungen}


\section{Ableitungsregeln II}


\section{Tangente/Normale}


\section{Implizit Ableiten Kreis}


\section{Höhere Ableitungen Exp}

