% coding:utf-8
Die Musterlösungen vom Testat 2 sind noch nicht überprüft haben und können daher noch Fehler enthalten! 
\section{Gleichung Sekante}
\[ A = (a_x | a_y) \]
\[ B = (b_x | b_y) \]
\[ y = \frac{b_y - a_y}{b_x - a_y} \cdot x + f(0) \]

\section{Gleichung Tangente}
\[ A = (a_x | a_y) \]
\[ f(x) = a x^2 + b x + c \]
\[ T(x) = f'(x_0)(x - x_0) + f(x_0) \]
\[ T(a_x) = (2 a x + b)(x - a_x) + a x^2 + b x + c \]
\[ T(a_x) = 2 a x^2 - 2 a x a_x + b x - b a_x + a x^2 + b x + c \]
\[ T(a_x) = 3 a x^2 - 2 a x a_x + 2 b x - b a_x + c \]

\section{Senkrechter Wurf}


\section{Schnitt mit Gerade}


\section{Intervall einer Funktion}
a) 
Die Funktion ist dort wachsend, wo die Ableitung $\geq 0$ ist. 
b) 
Die Funktion ist dort rechtsgekrümmt, wo die zweite Ableitung $\leq 0$ ist. 

\section{Unbekannte Funktion Ableitung}


\section{Diferenzierbarkeit}


\section{Ableitungsregeln I}


\section{Höhere Ableitungen}


\section{Ableitungsregeln II}


\section{Tangente/Normale}


\section{Implizit Ableiten Kreis}


\section{Höhere Ableitungen Exp}

