% coding:utf-8
\documentclass[a4paper,10pt,fleqn]{book}
 
\usepackage[utf8]{inputenc} % für input utf8
\usepackage[T1]{fontenc} %Schriftcodierung mit UTF-8
\usepackage{textcomp} %Erweiterung von fontenc
\usepackage{lmodern} %Erweiterung des Zeichensatzes

\usepackage{graphics}
\usepackage{graphicx}
\usepackage[ngerman]{babel}
\usepackage{amsmath}
%\usepackage[all]{xy}
%\usepackage[xindy]{glossaries}
\usepackage{makeidx}
\usepackage{pdfpages}
\usepackage{graphicx}
\usepackage{hyperref}
\usepackage{amssymb}

\DeclareMathOperator{\arccosh}{arccosh}

%\title{Musterlösungen Testate Mathematik}
%\author{Daniel Winz, Ervin Mazlagic, Adrian Imboden}
%\date{\today}

\begin{document}
\Huge
\vfill
\noindent
Die aktuelle Musterlösung befindet sich auf \\\\
\url{fosa.adinox.ch} \\\\
unter Testate. \\\\
\Large
Euer fosamath-Team


%\maketitle
%
%\chapter*{Über diese Arbeit}
%Dies ist das Ergebnis einer Zusammenarbeit auf Basis freier Texte erstellt von Studierenden der Fachhochschule Luzern. 
%
%Dieses Schriftstück ist lizenziert unter der GPLv2 und der \TeX-  bzw. \LaTeX- Code ist auf \url{github.com/daniw/fosamath} hinterlegt.
%
%
%\tableofcontents
%\chapter{Testat 1 - Vektorgeometrie}
%% coding:utf-8
\section{Vektoren}
\subsection{Aufgabe 1}
\[  \]
\[  \]
\[  \]

\subsection{Aufgabe 2}
\[  \]
\[  \]
\[  \]

\subsection{Aufgabe 3}
\[  \]
\[  \]
\[  \]

\section{Folgen}
\subsection{Aufgabe 1}
\[ a_n = a_1 + (n - 1)d \]
\[ d = a_{n+1} - a_n \]
\[ \frac{a_{n_2} - a_{n_1}}{n_2 - n_1} = d  \]
\[ \underline{\underline{a_{n_3} = a_{n_2} + d \cdot (n_3 - n_2) = a_{n_1} + d \cdot (n_3 - n_1)}} \]

\subsection{Aufgabe 2}
\[  \]
\[  \]
\[  \]

\subsection{Aufgabe 3}
\[  \]
\[  \]
\[  \]

\subsection{Aufgabe 4}
\[  \]
\[  \]
\[  \]

\section{Reihen}
\subsection{Aufgabe 1}
\[  \]
\[  \]
\[  \]

\subsection{Aufgabe 2}
\[  \]
\[  \]
\[  \]

\subsection{Aufgabe 3}
\[  \]
\[  \]
\[  \]

\subsection{Aufgabe 4}
\[  \]
\[  \]
\[  \]

%\chapter{Testat 2 - Folgen und Reihen}
%% coding:utf-8
Die Musterlösungen vom Testat 2 sind teilweise noch nicht überprüft und können daher noch Fehler enthalten! \\
Überprüfte Lösungen sind an der doppelten Unterstreichung zu erkennen. 
\section{Gleichung Sekante}
\[ A = (a_x | a_y) \]
\[ B = (b_x | b_y) \]
\[ y = \frac{b_y - a_y}{b_x - a_y} \cdot x + f(0) \]

\section{Gleichung Tangente}
\[ A = (a_x | a_y) \]
\[ f(x) = a x^2 + b x + c \]
\[ T(x) = f'(x_0)(x - x_0) + f(x_0) \]
\[ T(a_x) = (2 a x + b)(x - a_x) + a x^2 + b x + c \]
\[ T(a_x) = 2 a x^2 - 2 a x a_x + b x - b a_x + a x^2 + b x + c \]
\[ T(a_x) = 3 a x^2 - 2 a x a_x + 2 b x - b a_x + c \]

\section{Senkrechter Wurf}


\section{Schnitt mit Gerade}


\section{Intervall einer Funktion}
a) 
Die Funktion ist dort wachsend, wo die Ableitung $\geq 0$ ist. \\
b) 
Die Funktion ist dort rechtsgekrümmt, wo die zweite Ableitung $\leq 0$ ist. 

\section{Unbekannte Funktion Ableitung}
\[ f(x) = \frac{u(x) + a}{u(x) + b} \]
\[ u(0) = c \]
\[ u'(0) = d \]
\[ \underline{\underline{f(0) = \frac{c + a}{c + b}}} \]
\[ f'(0) = \frac{(d) \cdot (c + b) - (c + a) \cdot (d)}{(d + b)^2} \]
\[ f'(0) = \frac{(c d + b d) - (c d + a d)}{(c + b)^2} \]
\[ f'(0) = \frac{c d + b d - c d - a d}{(c + b)^2} \]
\[ \underline{\underline{f'(0) = \frac{b d - a d}{(c + b)^2}}} \]

\section{Differenzierbarkeit}
\[ f_1(x) = \frac{x + a}{a \cdot x} \quad x \leq -1 \]
\[ f_2(x) = \frac{b \cdot x}{x + 2 a} \quad x > -1 \]
\[ f_1(c) = f_2(c) \]
\[ f_1'(c) = f_2'(c) \]
\[ \frac{x + a}{a \cdot x} = \frac{b \cdot x}{x + 2 a} \]
\[ \frac{(x + a)(x + 2 a)}{a \cdot x} = b \cdot x \]
\[ \frac{(x + a)(x + 2 a)}{a \cdot x^2} = b \]

\[ \frac{1 \cdot a \cdot x - (x + a) \cdot a}{(a x)^2} = \frac{b \cdot (x + 2 a) - b x \cdot 1}{(x + 2 a)^2} \]
\[ \frac{a x - a x - a^2}{(a x)^2} = \frac{b x + 2 a b - b x}{(x + 2 a)^2} \]
\[ -\frac{a^2}{(a x)^2} = \frac{2 a b}{(x + 2 a)^2} \]
\[ -\frac{1}{x^2} = \frac{2 a b}{(x + 2 a)^2} \]
\[ -\frac{1}{x^2} = \frac{2 a \frac{(x + a)(x + 2 a)}{a \cdot x^2}}{(x + 2 a)^2} \]
\[ -\frac{1}{x^2} = \frac{2 a \frac{(x + a)}{a \cdot x^2}}{(x + 2 a)} \]
\[ -1 = \frac{2 (x + a)}{(x + 2 a)} \]
\[ -(x + 2 a) = 2 (x + a) \]
\[ -x - 2 a = 2 x + 2 a \]
\[ -3 x = 4 a \]
\[ \underline{\underline{a = -\frac{3}{4} x}} \]
\[ b = \frac{(x + a)(x + 2 a)}{a \cdot x^2} \]
\[ b = \frac{(x -\frac{3}{4} x)(x - \frac{6}{4} x)}{-\frac{3}{4} x \cdot x^2} \]
\[ b = -\frac{(\frac{1}{4} x)(- \frac{1}{2} x)}{\frac{3}{4} x \cdot x^2} \]
\[ b = \frac{\frac{1}{8} x^2}{\frac{3}{4} x \cdot x^2} \]
\[ \underline{\underline{b = \frac{1}{6 x} }} \]

\section{Ableitungsregeln I}


\section{Höhere Ableitungen}


\section{Ableitungsregeln II}


\section{Tangente/Normale}


\section{Implizit Ableiten Kreis}


\section{Höhere Ableitungen Exp}


%\chapter{Testat 3 - Differenzialrechnung}
%% coding:utf-8
\section{Polynomfunktion}
Punkt mit lokalem Extremwert: $(e,g)$\\
Punkt mit Wendepunkt: $(h,i)$
\[f(x) = a \cdot x^3 + b \cdot x^2 + c \cdot x + d\]
\[f'(x) = 3 \cdot a \cdot x^2 + 2 \cdot b \cdot x + c\]
\[f''(x) = 6 \cdot a \cdot x + 2 \cdot b\]
\[f'''(x) = 6 \cdot a\]
\[f(e) = g\]
\[f'(e) = 0\]
\[f''(e) \neq 0\]
\[f(h) = i\]
\[f''(h) = 0\]
\[f'''(h) \neq 0\]
Solve mit TR:
\[\text{solve}\left(\begin{matrix}
a \cdot e^3 + b \cdot e^2 + c \cdot e + d = g\\
3 \cdot a \cdot e^2 + 2 \cdot b \cdot e + c = 0\\
6 \cdot a \cdot e + 2 \cdot b \neq 0\\
a \cdot h^3 + b \cdot h^2 + c \cdot h + d = i\\
6 \cdot a \cdot x + 2 \cdot b\\
6 \cdot a \neq 0
\end{matrix} \quad ,a,b,c,d\right)\]
\[f(x) = a \cdot x^3 + b \cdot x^2 + c \cdot x + d\]

\section{Kettenlinie}
Aufgabenstellung: 
\[y(x) = a \cdot \cosh\left(\frac{x}{c}\right) + b\]
a)\\
\[h = H - y(x) = y(\ell) - y(x)\]
\[h = a \cosh\left(\frac{\ell}{c}\right) + b - \left(a \cosh\left(\frac{x}{c}\right) + b\right)\]
\[h = a \cdot \cosh\left(\frac{\ell}{c}\right) - a \cdot \underbrace{\cosh\left(\frac{x}{c}\right)}_1\]
\[\rightarrow x = 0\]
\[\Rightarrow h_{max} = a \cdot \cosh\left(\frac{\ell}{c}\right)-a\]
\[\underline{\underline{h(\ell) = a \cosh\left(\frac{\ell}{c}\right) - a}}\]
b)
\[h(\ell_1) =a \cdot \cosh\left(\frac{\ell}{c}\right) - a\]
\[\cosh\left(\frac{\ell}{c}\right) = \frac{h(\ell_1 + a}{a}\]
\[\Rightarrow \underline{\underline{\arccosh\left(\frac{h(\ell_1 + a}{a}\right) \cdot a}}\]
c)
\[m(\ell) = h'(\ell) = \sinh\left(\frac{\ell}{c}\right)\]
\[m(\ell) = \tan(\beta) \quad\]
\[\Rightarrow \underline{\underline{\alpha = \frac{\pi}{2} - \beta = \frac{\pi}{2} - \arctan(m(\ell)) = \frac{\pi}{2} - \arctan\left(\sinh\left(\frac{\ell}{c}\right)\right)}}\]

\section{Maximale Fläche}
Aufgabenstellung: \\
\includegraphics[width=0.8\textwidth]{bilder/maximale_flaeche.pdf}
\[a \cdot x^2 + b \cdot y^2 = c\]
Lösung:
\[A = 4 \cdot A_1\]
\[A_1 = x \cdot f(x)\]
\[\Rightarrow A = 4 \cdot x \cdot f(x)\]
% \[f(x) \stackrel{?}{=} \]
\[c = a \cdot x^2 + b \cdot y^2\]
\[\Rightarrow y = \left( \frac{c - a \cdot x^2}{b} \right)^{\frac{1}{2}}\]
\[A = 4 \cdot x \cdot \left( \frac{c - a \cdot x^2}{b} \right)^{\frac{1}{2}}\]
Wo ist $A_1$ maximal? \\
$\rightarrow$ Dort wo die Ableitung $0$ ergibt. 
\[A' \stackrel{!}{=} 0\]
\[A' = 4 \left(\frac{c - a \cdot x^2}{b}\right)^\frac{1}{2} + 4 \cdot x \left(\frac{1}{2} \cdot \left(\frac{c - a \cdot x^2}{b}\right)^{-\frac{1}{2}} \cdot \left(\frac{-2 \cdot a \cdot x}{b}\right)\right) = 0\]
\[A' = \left(\frac{c - a \cdot x^2}{b}\right)^{\frac{1}{2}} + x \cdot \left(\frac{1 \cdot \left(\frac{-2ax}{b}\right)}{2 \cdot \left(\frac{c - ax^2}{b}\right)^{\frac{1}{2}}}\right) = 0\quad \left|\text{mit }\left(\frac{c - ax^2}{b}\right)^{\frac{1}{2}} \right.\text{ erweitern}\]
\[A' = \frac{c - ax^2}{b} + x \cdot \frac{-ax}{b} = 0\]
\[A' = c - ax^2 - ax^2 = c - 2 \left(ax^2\right) = 0\]
\[c = 2 \left(ax^2\right)\]
\[\underline{x = \pm\sqrt{\frac{c}{2a}}}\]
$\rightarrow$ Weil die Fläche nur positiv sein kann, gilt nur $x \leq 0$
\[\Rightarrow A = 4 x \cdot f(x) = 4 x \left(\frac{c - ax^2}{b}\right)^{\frac{1}{2}}\]
\[A = 4 \sqrt{\frac{c}{2a}} \left(\frac{c - a \sqrt{\frac{c}{2a}}^2}{b}\right)^{\frac{1}{2}}\]
\[A = 4 \sqrt{\frac{c}{2a} \left(\frac{c - a \frac{c}{2a}}{b}\right)}\]
\[\underline{\underline{A = 2 \sqrt{\frac{c^2}{ab} } = \frac{2c}{\sqrt{ab}}}}\]

\section{Statue}
Aufgabenstellung: \\
\includegraphics[width=0.8\textwidth]{bilder/statue.pdf}
\[ b = \text{Erde zu Statuenfuss} \]
\[ a = \text{Satuenfuss zu Statuenkopf} \]
\[ d = \text{Statue zu Betrachter} \]
\[ \alpha = \text{Winkel von Statuenkopf zu Statuenfuss} \]
Wo ist der Winkel $\alpha$ maximal? Dort wo die Ableitung der Funktion $\alpha(d)$ Null ergibt also $\alpha'(d)=0$ ist. 
Um dies zu bestimmen muss $\alpha$ definiert werden. 
Da dies auf Anhieb nicht möglich ist, kann man sich folgende Überlegung machen:
\[ \beta = \text{Winkel Betrachter zu Statuenboden} \]
\[ \gamma = \text{Winkel Betrachter zu Statuenkopf} \]
\[ \Rightarrow \gamma = \alpha + \beta \]
\[ tan(\gamma) = \tan(\alpha + \beta) = \left(\frac{a+b}{d}\right) \]
\[ \Rightarrow \alpha + \beta = arctan\left(\frac{a+b}{d}\right) \]
Nun haben wir eine neue Unbekannte $\beta$. Diese muss eliminiert bzw. substituiert werden durch etwas bekanntes oder gesuchtes.
\[ \tan(\beta) = \left(\frac{b}{d}\right) \rightarrow \beta = arctan\left(\frac{b}{d}\right) \]
\[ \Rightarrow \alpha = arctan\left(\frac{a+b}{d}\right) - \beta = arctan\left(\frac{a+b}{d}\right) - arctan\left(\frac{b}{d}\right) \]
\[ \alpha' \stackrel{!}{=} 0 \rightarrow \alpha' = \frac{-(a+b)}{d^2 + (a+b)^2} + \frac{b}{d^2 + b^2} = 0 \]
\[ \Rightarrow d = \sqrt{ab + b^2} \]

\section{Implizites Ableiten}
\[ x^3 + y^3 - a \cdot x \cdot y = 0 \]
a)\\
x und y in gegebene Formel einsetzen. Diese muss 0 werden, damit der entsprechende Punkt auf der Kurve liegt. 
b)\\
\[ x^3 + y^3 - a x y = 0 \]
\[ y = y(x) \quad \text{$y$ mit $y(x)$ substituieren} \]
\[ \rightarrow x^3 + y(x)^3 - a x y(x) = 0 \]
Ableiten mit Kettenregel. $y(x)$ ist jeweils die innere Ableitung. 
\[ \frac{dy}{dx} = 3x^2 + 3y(x)^2 \cdot y'(x) - a\left(y(x) + x \cdot y'(x)\right) = 0 \]
\[ 3x^2 + 3y(x)^2 \cdot y'(x) - a \cdot y(x) - a \cdot x \cdot y'(x) = 0 \]
\[ 3y(x)^2 \cdot y'(x) - a \cdot x \cdot y'(x) = - 3x^2 + a \cdot y(x) \quad \left|- 3x^2 + a \cdot y(x)\right.\]
\[ y'(x) \cdot \left(3y(x)^2 \cdot - a \cdot x \right) = a \cdot y(x) - 3x^2 \quad |y'(x)\text{ ausklammern}\]
\[ y'(x) = \frac{a \cdot y(x) - 3x^2}{3y(x)^2 \cdot - a \cdot x} \quad |:3y(x)^2 \cdot - a \cdot x \]
\[ y' = \frac{a \cdot y - 3x^2}{3y^2 \cdot - a \cdot x} \quad |y(x) = y \]
c)\\
\[ T(x) = f'(x) \cdot (x - x_0) - f(x_0) \]
\[ T(x) =  \frac{a \cdot y_0 - 3x_0^2}{3y_0^2 \cdot - a \cdot x_0} \cdot (x - x_0) - (x_0^3 + y_0^3 - a \cdot x_0 \cdot y_0) \]
$x_0$ und $y_0$ aus Aufgabenstellung einsetzen. \\\\
{\Huge Achtung! \\Fehler in Lösung. \\Wenn du den Fehler findest: \\Mail an: \href{mailto:daniel.winz@stud.hslu.ch}{daniw}}\\
d)\\
\[ f'(0) = 0 \]
\[ \Rightarrow  \frac{a \cdot y - 3x^2}{3y^2 \cdot - a \cdot x} = 0 \]
\[ x^3 + y3 - a \cdot x \cdot y = 0 \]
2 Gleichungen, 2 Unbekannte: Solve mit TR. \\
\[ \Rightarrow x = a \cdot \frac{\sqrt[3]{2}}{3}\]
\[ \Rightarrow y = a \cdot \frac{\sqrt[3]{2^2}}{3}\]
%\chapter{Testat 4 - Integralrechnung}
%% coding:utf-8
\section{Stammfunktion}
a)\\
$x<0$: Steigung negativ\\
$x=0$: Steigung null\\
$x<0$ Steigung positiv\\\\
b)\\
$e^x$ ist abgeleitet sich selbst. $e^{-x}$ ist abgeleitet $-e^{-x}$. 

\section{Autofahrt}
\[ v(t) = a t^3 - b t^2 + c t \]
\[ s(t) = \int (v(t)) dt = \underline{\underline{\frac{a t^4}{4} - \frac{b t^3}{3} + \frac{c t^2}{2}}} \]

\section{Schwimmbecken}
\[ v(t) = a \cdot e^{-bt} \]
\[ V(t) = \int v(t) dt = \int \left(a \cdot e^{-bt}\right) dt = -\frac{a \cdot e^{-bt}}{b} + c \]
c ist die Integrationskonstante
\[ V(0) = 0 \]
\[ -\frac{a \cdot e^{0}}{b} + c = 0 \]
\[ -\frac{a}{b} + c = 0 \]
\[ c = \frac{a}{b} \]
\[ \rightarrow V(t) = -\frac{a \cdot e^{-bt}}{b} + \frac{a}{b} = -\frac{a}{b} \left( e^{-bt} - 1 \right) \]
\[ -V(t) \cdot \frac{b}{a} = e^{-bt} - 1 \]
\[ -V(t) \cdot \frac{b}{a} + 1 = e^{-bt} \]
\[ -bt = \ln\left(-V(t) \cdot \frac{b}{a} + 1\right) \]
\[ \underline{\underline{t = -\frac{\ln\left(-V(t) \cdot \frac{b}{a} + 1\right)}{b}}} \]

\section{Fläche durch Tangente}
\[ f(x) = x^3 + x^2 \]
\[ f'(x) = 3 x^2 + 2 x \]
\[ T(x) = f'(x_0)(x - x_0) + f(x_0) \]
\[ T(x) = (3 {x_0}^3 + 2 x_0)(x-x_0) + {x_0}^3 + {x_0}^2 \]
\[ T(x) = 3 x {x_0}^2 - 3 {x_0}^3 + 2 x x_0 - 2 {x_0}^2 + {x_0}^3 + {x_0}^2 \]
\[ T(x) = 3 x {x_0}^2 - 2 {x_0}^3 + 2 x x_0 - {x_0}^2 \]
\[ T(x) = x(3 {x_0}^2 + 2 x_0) - (2 {x_0}^3 + {x_0}^2) \]
\[ T(x) = 0 \]
\[ x_1 = \frac{2 {x_0}^3 + {x_0}^2}{3 {x_0}^2 + 2 x_0} \]
\[ A = A_{Gesamt} - A_\Delta \]
\[ A_\Delta = \frac{(x_0 - x_1) \cdot f(x_0)}{2} = \frac{(x_0 - x_1) \cdot ({x_0}^3 + {x_0}^2)}{2} \]
\[ A_{Gesamt} = \int_0^{x_0} (x^3 + x^2) dx = \left.\frac{x^4}{4} + \frac{x^3}{3} \right|_{0}^{x_0} = \frac{{x_0}^4}{4} + \frac{{x_0}^3}{3} \]
\[ A = A_{Gesamt} - A_\Delta = \underline{\underline{\frac{{x_0}^4}{4} + \frac{{x_0}^3}{3} - \frac{(x_0 - x_1) \cdot ({x_0}^3 + {x_0}^2)}{2}}} = -\frac{{x_0}^4}{4} + \frac{5 {x_0}^3}{6} - \frac{x_1 ({x_0}^3 + {x_0}^2)}{2} \]

\section{Eingeschlossener Flächeninhalt}
Aufgabenstellung: 
\[ f(x) = ax^2 - bax \]
\[ g(x) = ax \]
\[ a < 0 \]
Lösung: 
\[ f(x_0) = g(x_0) \quad x_0 \neq 0 \]
\[ a{x_0}^2 - bax_0 = ax_0 \]
\[ a{x_0}^2 - (b+1)ax_0 = 0 \]
\[ x_{0_{1,2}} = \frac{(b+1)a \pm \sqrt{(b+1)^2a^2}}{2a} = \frac{b+1)a \pm (b+1)a}{2a} = \frac{(b+1) \pm (b+1)}{2} \]
\[ x_{0_1} = \frac{2(b+1)}{2} = b+1 \]
\[ x_{0_2} = \frac{0}{2} = 0 \quad \Rightarrow \text{keine Lösung} \]
\[ F(x) = \int_0^{x_0} (ax^2 - (b+1)ax) dx = \left.\frac{ax^3}{3} - \frac{b(a+1)x^2}{2}\right|_0^{x_0} = \frac{a{x_0}^3}{3} - \frac{(b+1)a{x_0}^2}{2} \]
\[ = \frac{a{(b+1)}^3}{3} - \frac{a{(b+1)}^3}{2} = \frac{2a(b+1)^3}{6} - \frac{3a(b+1)^3}{6} = -\frac{a(b+1)^3}{6} \]
\[ \Rightarrow \underline{\underline{a = -\frac{6F}{(b+1)^3}}} \]

\section{Horizontale Linie}
\[ f(x) = x^2 \]
\[ g(x) = y_0 \]
\[ f^{-1}(y) = \sqrt{y} \]
\[ F_1 = \int_{y_1}^{y_0} f^{-1}(y) dy = \int_{y_1}^{y_0} \left(\sqrt{y}\right) dy = \left.\frac{2}{3} y^{\frac{3}{2}}\right|_{y_1}^{y_0} = \frac{2}{3}{y_0}^{\frac{3}{2}} - \frac{2}{3}{y_1}^{\frac{3}{2}} \]
\[ F_2 = \int_{0}^{y_1} f^{-1}(y) dy = \int_{0}^{y_1} \left(\sqrt{y}\right) dy = \left.\frac{2}{3} y^{\frac{3}{2}}\right|_{0}^{y_1} = \frac{2}{3}{y_1}^{\frac{3}{2}} \]
\[ F_1 = F_2 \]
\[ \frac{2}{3}{y_0}^{\frac{3}{2}} - \frac{2}{3}{y_1}^{\frac{3}{2}} = \frac{2}{3}{y_1}^{\frac{3}{2}} \]
\[ \frac{2}{3}{y_0}^{\frac{3}{2}} = \frac{4}{3}{y_1}^{\frac{3}{2}} \]
\[ {y_0}^{\frac{3}{2}} = 2 \cdot {y_1}^{\frac{3}{2}} \]
\[ \frac{{y_0}^{\frac{3}{2}}}{2} = {y_1}^{\frac{3}{2}} \]
\[ y_1 = \underline{\underline{\frac{y_0}{2^{\frac{2}{3}}} \approx \frac{y_0}{1.5874}}} \]

\section{Vase}
\[ f(x) = \sqrt{x+a} \]
\[ g(x) = \sqrt{x-b} \]
\[ V(x) = \pi \int_0^h f(x)^2 dx - \pi \int_b^h g(x)^2dx = \pi \left( \int_0^h (x+a) dx - \int_b^h (x-b) dx\right) \]
a)
\[ V(x) = \pi\left(\left. \frac{h^2}{2} + ah \right|_0^h - \left(\left. \frac{x^2}{2} - bx \right|_b^h \right) \right) = \pi \left(  \frac{h^2}{2} + ah - \left( \frac{h^2}{2} - bh - \frac{b^2}{b} + b^2 \right)\right) \]
\[ = \pi \left(\frac{h^2}{2} + ah - \frac{h^2}{2} + bh + \frac{b^2}{2} - b^2\right) = \pi \left(ah + bh + \frac{b^2}{2} - b^2\right) \]
\[ V(x) = \pi \left(ah + bh - b^2\right) \]
\[ \Rightarrow \underline{\underline{m = V \cdot \rho = \pi \cdot \rho \cdot \left(ah + bh - \frac{b^2}{2}\right)}} \]
b)
\[ A_{M_x} = 2 \pi \int_0^h f(x) \sqrt{1 + f'(x)^2} dx \]
\[ A_{M_x} = 2 \pi \int_0^h \sqrt{x + a} \cdot \sqrt{1 + \left( \frac{1}{2 \sqrt{x + a}} \right)^2} dx \]
\[ A_{M_x} = 2 \pi \int_0^h \sqrt{(x + a) \cdot \left(1 + \frac{1}{4 (x + a)} \right)} dx \]
\[ A_{M_x} = 2 \pi \int_0^h \sqrt{x + a + \frac{1}{4}} dx \]
\[ A_{M_x} = 2 \pi \int_0^h \frac{1}{2} \sqrt{4x + 4a + 1} dx \]
\[ A_{M_x} = \left. 2 \pi \cdot \frac{1}{2} \frac{2}{3}\left( 4x + 4a + 1 \right)^{\frac{3}{2}} \cdot \frac{1}{4} \right|_0^h \]
\[ A_{M_x} = \left. \frac{1}{6} \pi \left(4x + 4a + 1\right)^{\frac{3}{2}} \right|_0^h \]
\[ \rightarrow A_{M_x} = \frac{1}{6} \pi \left(\left( 4h + 4a + 1 \right)^{\frac{3}{2}} - \left(4a + 1\right)^{\frac{3}{2}}\right) \]
\[ A_B = f(0)^2 \cdot \pi = a \pi \]
\[ A_D = f(h)^2 \cdot \pi - g(h)^2 \cdot \pi = \pi \left( f(h)^2 - g(h)^2 \right) = \pi \left( h + a - h + b \right) = \pi \left( a + b \right) \]
\[ A =  A_{M_x} + A_B + A_D = \frac{1}{6} \pi \left(\left( 4h + 4a + 1 \right)^{\frac{3}{2}} - \left(4a + 1\right)^{\frac{3}{2}}\right) + a \pi + \pi \left( a + b \right) \]
\[ A =  A_{M_x} + A_B + A_D = \pi \left(\frac{1}{6} \left(\left( 4h + 4a + 1 \right)^{\frac{3}{2}} - \left(4a + 1\right)^{\frac{3}{2}}\right) + a + \left( a + b \right)\right) \]
\[ \underline{\underline{A =  A_{M_x} + A_B + A_D = \pi \left(\frac{1}{6} \left(\left( 4h + 4a + 1 \right)^{\frac{3}{2}} - \left(4a + 1\right)^{\frac{3}{2}}\right) + 2a + b\right)}} \]

\section{Zykloide}
\[ x(t) = a \cdot (t - \sin(t) \]
\[ y(t) = b \cdot (1 - \cos(t)) \]
\[ \dot{x}(t) = a - a \cos(t) \]
\[ \dot{y}(t) = b \sin(t) \]
Folgend wird nur der Fall $a = b$ behandelt. In den bisherigen Versuchen war dies immer der Fall. \\Sollte dies einmal nicht der Fall sein $\rightarrow$ Bitte Nachricht an uns. Das wird dann ergänzt. 
\[ S = \int_0^{2\pi} \sqrt{\dot{x}^2 + \dot{y}^2} dt = \int_0^{2\pi} \sqrt{(a - a \cos(t))^2 + (a \sin(t))^2} dt \]
\[ S = \int_0^{2\pi} \sqrt{a^2 \cdot (1 - \cos(t))^2 + a^2 \sin(t)^2} dt = a \cdot \int_0^{2\pi} \sqrt{(1 - \cos(t))^2 + \sin(t)^2} dt \]
\[ S = a \cdot \int_0^{2\pi} \sqrt{1 - 2 \cos(t) + \underbrace{\cos(t)^2 + \sin(t)^2}_1} dt = a \cdot \int_0^{2\pi} \sqrt{2 - 2 \cos(t)} dt \]
Mit Taschenrechner auflösen: 
\[ \underline{\underline{\Rightarrow S = 8 \cdot a}} \]

\section{Fahrzeug}
\includegraphics[width=0.8\textwidth]{bilder/fahrzeug_svg.pdf}
a)
\[ s = s_1 + s_2 + s_3 \]
\[ s_1 = \int_0^{t_1} v(t) dt = \int_0^{t_1} \left( \frac{v_{max}}{t_1} \cdot t \right) dt = \left. \frac{v_{max} \cdot t^2}{2 t_1} \right|_0^{t_1} = \frac{v_{max} \cdot {t_1}^2}{2 t_1} = \frac{v_{max} \cdot {t_1}}{2} \]
\[ s_2 = \int_{t_1}^{t_2} v_{max} dt = t_2 \cdot v_{max} - t_1 \cdot v_{max} = v_{max} \cdot (t_2 - t_1) = v_{max} \cdot t_{1,2} \]
\[ s_3 = \int_{t_2}^{t_3} \left(-\frac{v_{max}}{t_{2,3}} \cdot (t - t_{0,1} - t_{1,2} - t_{2,3})\right) dt = \int_{t_2}^{t_3} \left(-\frac{v_{max}}{t_{2,3}} \cdot (t - t_{0,3})\right) dt \]
\[ s_3 = -\frac{v_{max}}{t_{2,3}} \cdot \int_{t_2}^{t_3} \left(t - t_3\right) dt = -\frac{v_{max}}{t_{2,3}} \cdot \left.\left(\frac{t^2}{2} - t \cdot t_3\right)\right|_{t_2}^{t_3} \]
\[ s_3 = -\frac{v_{max}}{t_{2,3}} \cdot \left(\frac{{t_3}^2}{2} - {t_3}^2 - \frac{{t_2}^2}{2} + t_2 \cdot t_3\right) = -\frac{v_{max}}{t_{2,3}} \cdot \left(-\frac{{t_3}^2}{2} - \frac{{t_2}^2}{2} + t_2 \cdot t_3\right) \]
\[ s_3 = -\frac{v_{max}}{t_{2,3}} \cdot \left(-\frac{(t_{2,3} - t_2)^2}{2} - \frac{{t_2}^2}{2} + t_2 \cdot (t_{2,3} - t_2)\right) \]
\[ s_3 = -\frac{v_{max}}{t_{2,3}} \cdot \left(-\frac{{t_{2,3}}^2 - t_{2,3} t_2 + {t_2}^2}{2} - \frac{{t_2}^2}{2} + t_2 t_{2,3} + {t_2}^2\right) \]
\[ s_3 = -\frac{v_{max}}{t_{2,3}} \cdot \left(-\frac{{t_{2,3}}^2}{2} + \frac{t_{2,3} t_2}{2} - \frac{{t_2}^2}{2} - \frac{{t_2}^2}{2} + t_2 t_{2,3} + {t_2}^2\right) \]
\[ s_3 = -\frac{v_{max}}{t_{2,3}} \cdot \left(-\frac{{t_{2,3}}^2}{2} + \frac{t_{2,3} t_2}{2} + t_2 t_{2,3}\right) \]
\[ s_3 = -v_{max} \cdot \left(-\frac{{t_{2,3}}}{2} + \frac{t_2}{2} + t_2\right) \]
Fehler in der Berechnung von $s_3$. Korrekt wäre: 
\[ s_3 = \frac{v_{max} \cdot t_{2,3}}{2} \]
\[ s = s_1 + s_2 + s_3 = \frac{v_{max} \cdot {t_1}}{2} + v_{max} \cdot t_{1,2} + \frac{v_{max} \cdot t_{2,3}}{2} \]
\[ \underline{\underline{s = v_{max} \cdot \left(\frac{{t_1}}{2} + t_{1,2} + \frac{t_{2,3}}{2}\right)}} \]
b)
\[ \underline{\underline{\overline{v} = \frac{s}{t_3} = \frac{s}{t_{1} + t_{1,2} + t_{2,3}}}} \]
c)
\[ v(t_{0_1}) = v(t_{0_2}) = \overline{v} \]
\[ \rightarrow \frac{v_{max}}{t_1} \cdot t_{0_1} = \frac{s}{t_3} \]
\[ \underline{\underline{\Rightarrow t_{0_1} = \frac{s \cdot t_1}{t_3 \cdot v_{max}}}} \]
\[ \rightarrow -\frac{v_{max}}{t_{2,3}} \cdot (t_{0_2} - t_3) = \frac{s}{t_3} \]
\[ t_{0_2} - t_3 = -\frac{s \cdot t_{2,3}}{t_3 \cdot v_{max}} \]
\[ \underline{\underline{\Rightarrow t_{0_2} = -\frac{s \cdot t_{2,3}}{t_3 \cdot v_{max}} + t_3 = -\frac{s \cdot t_{2,3}}{t_3 \cdot v_{max}} + t_1 + t_{1,2} + t_{2,3}}} \]

\end{document}